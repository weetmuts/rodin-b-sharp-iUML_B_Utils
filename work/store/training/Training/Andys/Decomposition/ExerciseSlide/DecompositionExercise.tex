%\documentclass[handout]{beamer}
\documentclass{beamer}

\usepackage{bsymb}
\usepackage{b}
\usepackage{xcolor}

% Purpose of modelling?

% difference between axioms and invariants?
% can macine have axioms and vice versa

% why state invariants explicitly
% explain keywords

% ctr := ctr-1  versus  ctr=ctr-1

% explain skip

\mode<presentation>
{
    	%\usetheme{Warsaw}
	\setbeamertemplate{footline}
	{\centerline{\insertframenumber/\inserttotalframenumber}}
} 


\title{Decomposition Exercise}

\author{Andy Edmunds}

\institute{ University of Southampton }



\begin{document}



\begin{frame}

\titlepage

\end{frame}



\begin{frame}

\frametitle{Decomposition}
\begin{itemize}
\item Create/use an abstract model of a credit/debit instruction passing through middleware, to a target system.
\item The instruction at source \emph{src} has a value, and a command i.e. credit or debit.
\item Model the transfer of the information to the target.
\item Add parameters to send and receive event to faciitate decomposition. 
\item Decompose using Shared Event Decomposition.
\end{itemize}

\begin{itemize}
\item Try the same with Shared Variable decompostion and compare.
\end{itemize}

\begin{figure}
	\includegraphics[scale=0.5]{Overview}
\end{figure}
\end{frame}





\end{document}




